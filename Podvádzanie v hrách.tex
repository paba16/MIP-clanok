% Metódy inžinierskej práce

\documentclass[10pt,twoside,slovak,a4paper]{article}

\usepackage[slovak]{babel}
%\usepackage[T1]{fontenc}
\usepackage[IL2]{fontenc} % lepšia sadzba písmena Ľ než v T1
\usepackage[utf8]{inputenc}
\usepackage{graphicx}
\usepackage{url} % príkaz \url na formátovanie URL
\usepackage{hyperref} % odkazy v texte budú aktívne (pri niektorých triedach dokumentov spôsobuje posun textu)

\usepackage{cite}
%\usepackage{times}

\pagestyle{headings}

\title{Podvádzanie v hrách\thanks{Semestrálny projekt v predmete Metódy inžinierskej práce, ak. rok 2022/23, vedenie: Vladimír Mlynarovič}} % meno a priezvisko vyučujúceho na cvičeniach

\author{Patrik Baran\\[2pt]
	{\small Slovenská technická univerzita v Bratislave}\\
	{\small Fakulta informatiky a informačných technológií}\\
	{\small \texttt{xbaranp@stuba.sk}}
	}

\date{\small 11. október 2022} % upravte



\begin{document}

\maketitle

\begin{abstract}
Slovo podvádzanie má v našej spoločnosti negatívny význam. Je na to aj dobrý dôvod, keďže zvyčajne vedie ku neférovej výhode a často býva aj trestné. Pri hrách však ľudia pravidelne vykonávajú činnosti, ktoré nie sú v súlade s pôvodnými zámermi herných dizajnérov.

Najviac príležitostí stretnutia s podvádzaním máme v prípadoch, keď spôsobí neférovú výhodu v hre hranej viacerými ľuďmi naraz. Podvádzanie nie je možné iba v hrách pre viacerých, často prebieha práve keď sme vo svete hry sami.

V rámci tohto článku sa zamýšľame nad rozsahom pojmu podvádzanie. Potom ako tento rozsah zadefinujeme sa pokúsime poskytnúť dôvody prečo niektorí hráči podvádzajú, vysvetliť rozdiely podvádzania pri hrách pre jedného a hrách pre viac hráčov a vysvetliť ako mu môžu herný vývojári zabrániť.
\end{abstract}

\newpage


\section{Úvod}

Podvádzanie v hrách je konanie, ktoré v hrách prebieha už od ich samého počiatku. Nie vždy je však presne zadefinované čo všetko pod tento pojem spadá, teda či záleží na názore ľudí hrajúcich hru, či ma názore ľudí ktorí ju vytvorili. V tomto článku sa budeme zamýšľať nad problematikou podvádzania v hrách.

Slovo podvádzanie v našom jazyku nesie mnoho významov, môže znamenať neveru, trik, získanie neférovej výhody. Kým z týchto významov je zrejmé prečo sa podvádzanie chápe ako niečo čisto negatívne, vo svete hier to vôbec nemusí byť tak, nakoľko sa môže odohrať aj mimo sociálnych prostredí.

\section{Hranice}

Problém s presným definovaním pokiaľ sa jedná o akceptovateľné správanie a odkiaľ sa jedná o podvádzanie spočíva v subjektívnosti. Je samozrejmé, že každý človek by rád zahrnul svoje správanie ešte pred hranicou podvádzania.

Jedna z možnosí ako tento problém vyriešiť je výber charakteristiky podvádzania a následné označenie všetkých aktivít spojených s hraním, ktoré spĺňajú danú charakteristiku, za podvádzanie. Pre účely tohto článku sme sa rozhodli definovať podvádzanie ako každá výhoda, ktorá sa nedá dosiahnuť čisto hraním a spôsobom, aký mal pôvodný herný dizajnér na mysli.

\subsection{Herné mechaniky}
\begin{itemize}
\item V hrách sa občas nedopatrením objaví nejaká chyba, ktorá umožni hráčovi prejsť cez pevné objekty, získať neobmedzené množstvo životov, peňazí a podobne. Takéto chyby avšak nepatria ku plánovanému správaniu. To znamená že to nepatrí ku zamýšľanému spôsobu hrania hry, teda jedná sa o podvádzanie

\item Na druhej strane však môže nastať prípad, keď zámerná herná mechanika sa dá využiť v neočakávanom čase namiesto predpokladaného riešenia na preskočenie problému. V takomto prípade si hráč ani len nemusí uvedomiť, že niečo preskočil. Napriek tomu získal výhodu tým, že nehral zmýšľaným spôsobom, takže podvádzal.
\end{itemize}

\subsection{Pomoc}
\begin{itemize}
\item Návody nezvyknú byť súčasťou základnej hry, a preto ich považujeme za podvádzanie
\item Na druhej strane nápovedy sa v niektorých hrách zvyknú vyskytovať. Z nášho pohľadu však dizajn úrovní sa začína vytvorením samotnej úrovne, následne nastane otestovanie úrovne, a nakoniec (ak sa v danej hre vôbec nachádzajú) sa na náročných miestach pridajú nápovedy. Nakoľko nie sú potrebné na dokončenie úrovne, považujeme ich za niečo presahujúce predpísané pravidla hry. Už to nie je čisto hranie hry, čo znamená že ich považujeme za podvádzanie
\item Požiadanie iného človeka o pomoc sa môže udiať v hre, no ak sa nemá počítať medzi nápovedy, tak jediná možnosť je, keby danú čásť odohral za vás. Toto znamená že vy osobne nebudete hrať (alebo ovládať celú hru), čo je podvádzanie.
\end{itemize}

\subsection{Nákupy v hre}
\begin{itemize}
\item Nákupy v hrách sa do hier vkladajú s úmyslom zarobiť peniaze na vývoj ďaľších hier. Na to aby presvedčili hráča minúť v danej hre peniaze zvyčajne ponúkajú možnosť nakúpiť rôzne herné suroviny. Aj keď existujú povolania ktoré umožňujú zarobiť peniaze hraním, vo väčšine prípadov tomu tak nie je, a teda sa jedná o neférovu výhodu, ide o využívanie suroviny mimo hry na získanie výhody v hre.
\end{itemize}

\subsection{Úprava herných súborov}
\begin{itemize}
\item Cheaty sú externý softvér špecificky vytvorený na uľahčenie hry. Drasticky menia považujeme za podvádzanie.
\item Bežná metóda úpravy herných súborov je pridanie rôznych módov. Tieto módy môžu hru uľahčit, sťažiť, či len jednoducho rozšíriť. Množstvo hier takéto módovanie aj priamo podporuje. V tomto prípade sa je potrebné zamyslieť či sa nejedná o úplne inú hru/ herní mód, teda či je opravenú a pôvodnú verziu možné porovnať s ohľadom na spôsoby hrania.
\item Úprava obsahu súboru pôvodnej hry spôsobuje zvyčajne zmenu hodnoty naplánovanej herným dizajnérom, a teda ide o podvádzanie.
\end{itemize}

Koniec koncov, vždy záleží na jednotlivých hráčoch, respektívne komunitách v ktorých hrajú. Rozdiely medzi komunitami bývajú obrovské. Napríklad pri komunitách zameraných na speedruny, teda dokončenie hry v najkratšom možnom čase, siahajú pravidlá od povolenia iba pôvodnej verzie hry na pôvodnom hernom systéme, kým pri ostatných je úplne v pohode napríklad aj využívanie externých programov, ktoré načítavajú údaje priamo z hry.

\section{Motivácie}%?

Predchádzajúci výskum objavil 4 hlavné dôvody, prečo sa môže hráč rozhodnúť podvádzať: pocit zaseknutia, túžba zahrať sa na boha, nuda a snaha správať sa ako somár.\cite{mood} Výskumov s podobným cieľom bolo už viacero s podobnými odpoveďami. Napríklad v článku [pridať do literatúry] sa spomína porušenie úprimnosti v mobilných hrách. Samozrejme že dôvodom prečo nespomenuli chýbajúcu dôveru v hru môže byť aj iná definícia podvádzania.


% moralita p2w, drak v Elden ringu

% odkaz na článok, ktorý potvrdzuje motivácie podvádzania z viacerých zdrojov



\section{Opatrenia proti podvádzaniu}

Množstvo online hier v dnešnej dobe používa rôzne systémy na detekciu podvádzania, ktoré sa spúšťajú priamo s hrou. Pri hrách pre jedného je v dnešnej dobe trend vyžadovať internetové spojenie na overenie hry.
\section{Záver} \label{zaver} % prípadne iný variant názvu



%\acknowledgement{Ak niekomu chcete poďakovať\ldots}


% týmto sa generuje zoznam literatúry z obsahu súboru literatura.bib podľa toho, na čo sa v článku odkazujete
\bibliography{literatura}
\bibliographystyle{plain} % prípadne alpha, abbrv alebo hociktorý iný
\end{document}
