% Metódy inžinierskej práce

\documentclass[10pt, oneside, slovak,a4paper]{article}

\usepackage[slovak]{babel}
%\usepackage[T1]{fontenc}
\usepackage[IL2]{fontenc} % lepšia sadzba písmena Ľ než v T1
\usepackage[utf8]{inputenc}
\usepackage{graphicx}
\usepackage{url} % príkaz \url na formátovanie URL
\usepackage{hyperref} % odkazy v texte budú aktívne (pri niektorých triedach dokumentov spôsobuje posun textu)

\usepackage{cite}
%\usepackage{times}

\pagestyle{headings}

\title{Podvádzanie v hrách\thanks{Semestrálny projekt v predmete Metódy inžinierskej práce, ak. rok 2022/23, vedenie: Vladimír Mlynarovič}} % meno a priezvisko vyučujúceho na cvičeniach

\author{Patrik Baran\\[2pt]
	{\small Slovenská technická univerzita v Bratislave}\\
	{\small Fakulta informatiky a informačných technológií}\\
	{\small \texttt{xbaranp@stuba.sk}}
	}

\date{\small 11. október 2022} %



\begin{document}

\maketitle

\begin{abstract}
Slovo podvádzanie má v našej spoločnosti negatívny význam. Je na to aj dobrý dôvod, keďže zvyčajne vedie ku neférovej výhode a často býva aj trestné. Pri hrách však ľudia pravidelne vykonávajú činnosti, ktoré nie sú v súlade s pôvodnými zámermi herných dizajnérov.

Najviac príležitostí stretnutia s podvádzaním máme v prípadoch, keď spôsobí neférovú výhodu v hre hranej viacerými ľuďmi naraz. Podvádzanie nie je možné iba v hrách pre viacerých, často prebieha práve keď sme vo svete hry sami.

V rámci tohto článku sa zamýšľame nad rozsahom pojmu podvádzanie, so zameraním na počítačové hry. Potom ako tento rozsah zadefinujeme sa pokúsime poskytnúť dôvody prečo niektorí hráči podvádzajú, vysvetliť rozdiely podvádzania pri hrách pre jedného a hrách pre viac hráčov a vysvetliť ako mu môžu herný vývojári zabrániť.
\end{abstract}

\newpage


\section{Úvod}

Podvádzanie je konanie, ktoré v hrách prebieha už od ich samého počiatku. Spôsobov ako podvádzať je mnoho, možno 
%
%content opísať podvádzanie v stolných hrách, vážené kocky, ...., peeking in hide and seek, =: viedlo ku neferovej vyhode...
%
Nie vždy je však presne zadefinované čo všetko pod tento pojem spadá, teda či záleží na názore ľudí hrajúcich hru, či na názore ľudí ktorí ju vytvorili. V tomto článku sa budeme zamýšľať nad problematikou podvádzania v hrách.

Slovo podvádzanie v našom jazyku nesie mnoho významov, môže znamenať neveru, trik, získanie neférovej výhody. Kým z týchto významov je zrejmé prečo sa podvádzanie chápe ako niečo čisto negatívne, vo svete hier to vôbec nemusí byť tak, nakoľko sa môže odohrať aj mimo akéhokoľvek sociálneho prostredia.

% dobre prestudovana tema

% existential crisis - v hrách, vo video hrách
\section{Hranice}

Problém s presným definovaním pokiaľ siahajú hranice akceptovateľné správanie a odkiaľ sa jedná o podvádzanie spočíva v subjektívnosti. Je samozrejmé, že každý človek by rád zahrnul svoje správanie ešte pred hranicou podvádzania. % rozvinúť ďalej

Jedna z možností ako tento problém vyriešiť je výber charakteristiky podvádzania a následné označenie všetkých aktivít spojených s hraním, ktoré spĺňajú danú charakteristiku, za podvádzanie. Pre účely tohto článku sme sa rozhodli definovať podvádzanie ako každú výhodu, ktorú sa nedá dosiahnuť čisto hraním alebo spôsobom, aký mal pôvodný herný dizajnér na mysli.

\subsection{Herné mechaniky}
\begin{itemize}
\item V hrách sa občas nedopatrením objaví nejaká chyba, ktorá umožni hráčovi prejsť cez pevné objekty, získať neobmedzené množstvo životov, peňazí a podobne. Takéto chyby avšak nepatria ku plánovanému správaniu. To znamená že to nepatrí ku zamýšľanému spôsobu hrania hry, teda jedná sa o podvádzanie

\item Na druhej strane však môže nastať prípad, keď zámerná herná mechanika sa dá využiť v neočakávanom čase namiesto predpokladaného riešenia na preskočenie problému. V takomto prípade si hráč ani len nemusí uvedomiť, že niečo preskočil. Napriek tomu získal výhodu tým, že nehral zmýšľaným spôsobom, takže podvádzal.
\end{itemize}

\subsection{Pomoc}
\begin{itemize}
\item Návody nezvyknú byť súčasťou základnej hry, a preto ich považujeme za podvádzanie
\item Na druhej strane nápovedy sa v niektorých hrách zvyknú vyskytovať. Z nášho pohľadu však dizajn úrovní sa začína vytvorením samotnej úrovne, následne nastane otestovanie úrovne, a nakoniec (ak sa v danej hre vôbec nachádzajú) sa na náročných miestach pridajú nápovedy. Nakoľko nie sú potrebné na dokončenie úrovne, považujeme ich za niečo presahujúce predpísané pravidla hry. Už to nie je čisto hranie hry, čo znamená že ich považujeme za podvádzanie
\item Požiadanie iného človeka o pomoc sa môže udiať v hre, no ak sa nemá počítať medzi nápovede, tak jediná možnosť je, keby danú časť odohral za vás. Toto znamená že vy osobne nebudete hrať (alebo ovládať celú hru), čo je podvádzanie.
\end{itemize}

\subsection{Nákupy v hre}
\begin{itemize}
\item Nákupy v hrách sa do hier vkladajú s úmyslom zarobiť peniaze na vývoj ďaľších hier. Na to aby presvedčili hráča minúť v danej hre peniaze zvyčajne ponúkajú možnosť nakúpiť rôzne herné suroviny. Aj keď existujú povolania ktoré umožňujú zarobiť peniaze hraním, vo väčšine prípadov tomu tak nie je, a teda sa jedná o neférovú výhodu, ide o využívanie suroviny získanej mimo hry na získanie výhody v hre.

% pridať
\end{itemize}

\subsection{Úprava herných súborov}
\begin{itemize}
\item Cheaty sú externý softvér špecificky vytvorený na uľahčenie hry. Drasticky uľahčujú hru, a tak považujeme za podvádzanie.
\item Bežná metóda úpravy herných súborov je pridanie rôznych módov. Tieto módy môžu hru uľahčiť, sťažiť, či len jednoducho rozšíriť. Množstvo hier takéto módovanie aj priamo podporuje. V tomto prípade sa je potrebné zamyslieť či sa nejedná o úplne inú hru/ herní mód, teda či je opravenú a pôvodnú verziu možné porovnať s ohľadom na spôsob a ciele hrania.
\item Úprava obsahu súboru pôvodnej hry spôsobuje zvyčajne zmenu hodnoty naplánovanej herným dizajnérom, a teda ide o podvádzanie.
\end{itemize}

Koniec koncov, vždy záleží na jednotlivých hráčoch, respektíve komunitách v ktorých hrajú. Rozdiely medzi komunitami bývajú obrovské. Napríklad pri komunitách zameraných na speedruny, teda dokončenie hry v najkratšom možnom čase, siahajú pravidlá od povolenia iba pôvodnej verzie hry na pôvodnom hernom systéme, kým pri ostatných je úplne v pohode napríklad aj využívanie externých programov, ktoré načítavajú údaje priamo z hry.

\section{Motivácie}%?

Predchádzajúci výskum objavil 4 hlavné dôvody, prečo sa môže hráč rozhodnúť podvádzať: pocit zaseknutia, túžba zahrať sa na boha, nuda a snaha správať sa ako somár.\cite{mood} Výskumov s podobným cieľom bolo už viacero s podobnými odpoveďami. Samozrejme že dôvodom prečo nespomenuli chýbajúcu dôveru v hru môže byť aj iná definícia podvádzania.

Napríklad v článku [pridať do literatúry] sa spomína porušenie úprimnosti v mobilných hrách. Vysvetľuje, že koncový používatelia majú rôzne očakávania. Očakávajú napríklad súkromie svojich údajov, transparentnosť procesov v aplikácii a etické správanie platform a softvérových spoločností. [48]. Je zrejmé, že porušenie takýchto očakávaní má negatívny dopad na vzťah medzi hráčom a vývojármi tejto hry, a taktiež aj priamo na vzťah hráča ku hre. Takéto nerešpektovanie hráča z nášho pohľadu ho môže prinútiť, aby na odplatu prestal zase rešpektovať hru. Takéto vzájomné znevažovanie ale nemusí znamenať, že hráč túto hru prestane nadobro hrať. Druhou možnosťou je aj odveta voči samotnej hre........

Ďalšou možnosťou môže byť nejaká snaha o férovosť. Môže sa napríklad stať, že hra ponúka nejaký obsah za reálne peniaze. Na to, aby presvedčili niektorých hráčov na minutie peňazí v ich hre im spravia lákavú ponuku, ktorá im poskytne príležitosť preskočiť namáhavé získavanie surovín, šancu na získať lepšieho vybavenia v rulete a podobne. Hráč sa snaží dosiahnuť nejaký cieľ a podvádzanie v takomto prípade môže slúžiť na jednoduché dosiahnutie pocitu zadosťučinenia. No v prípadoch, keď každý hráč nie je izolovaný vo svojom svete a hráči medzi sebou aktívne súťažia, môže takáto príležitosť zmariť očakávanie férového súperenia. Najmä v prípadoch kde náhoda hrá svoju rolu môže hráč získať presvedčenie, že si zaslúži byť na vrchole rebríčka, čo znova môže viesť ku vyhľadávaniu neférovej výhody.

Jeden zaujímavý prípad príležitosti získať výhodu sa nachádza v hre Elden Ring. Jedná sa o veľmi náročnú, no férovú hru, kde máte možnosť počas svojho dobrodružstva naraziť ........

%drak v Elden ringu = otvorenie moralnej temy

% odkaz na článok, ktorý potvrdzuje motivácie podvádzania z viacerých zdrojov



\section{Opatrenia proti podvádzaniu}

Množstvo online hier v dnešnej dobe používa rôzne systémy na detekciu podvádzania, ktoré sa spúšťajú priamo s hrou. V dnešnej dobe je dokonca aj trend vyžadovať internetové spojenie aj v hre pre jedného na overenie hry.

Jasný spôsob ako odratiť hráčov od podvádzaní sú tresty. Tento spôsob je v dnešnej dobe celkom rozšírený, zvyčajne sa jedná o zákaz hrať na určitú dobu, až úplný zákaz. Jedná sa o formu inšpirovanú ozajstnými zákonmi, no existuje veľké množstvo iných spôsobos snáď lepšich než úplné zamietnutie práva hrať hru.

Jeden zo spôsobov ako odradiť hráčov od podvádzania môžu byť aj neočakávané dôsledky. 

Štúdia sa zamerala na efekt sledujúcich očí. Prišla s hypotézou, že virtuálna postava spôsobí pocit že nás niekto sleduje, teda starosť o našu reputáciu, čo znamená, že sa budeme väčšmi chcieť byť úprimný. \cite{not_alone} Túto hypotézu sa jej dokonca aj podarilo potvrdiť
% pouzitie moralnych dovodov na ... (tempting immoral actions with payoffs ?)

\section{Záver} 

 Najväčším problémom tejto témy je jej subjektívnosť. Nie je možné definovať podvádzanie jednoducho ako správanie vymykajúce sa zámerom herných dizajnérov. Definovanie ako získanie neférovej výhody však tiež nie je postačujúce, keď samotný systém hry poskytuje ostatným neférové výhody, alebo ich má nad vami. Aj keby sa to niekomu podarilo objektívne definovať podvádzanie, po niekoľkých rokoch by sa zmenilo vnímanie danej hry a definícia by prestala platiť, respektíve by prestala byť úplná.

 Dokonca aj hry pre jedného v dnešnej dobe zvyknú zaznamenávať určitý zoznam úspechov, čo informuje každého kto sa pozrie na náš profil o našich úspechoch, a teda aj do týchto hier prináša dilemy podvádzania v hrách pre viacerých hráčov.

Koniec koncov vždy záleží od ľudí riadiacich danú komunitu. Ak sa rozhodnú, že je niečo povolené bez súhlasu väčšíny, väčšina si môže založiť vlastnú frakciu s inými pravidlami.



\nocite{*}

% týmto sa generuje zoznam literatúry z obsahu súboru literatura.bib podľa toho, na čo sa v článku odkazujete
\bibliography{literatura}
\bibliographystyle{plain} % prípadne alpha, abbrv alebo hociktorý iný
\end{document}
