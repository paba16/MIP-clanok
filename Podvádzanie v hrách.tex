% Metódy inžinierskej práce

\documentclass[10pt, oneside, slovak,a4paper]{article}

\usepackage[slovak]{babel}
%\usepackage[T1]{fontenc}
\usepackage[IL2]{fontenc} % lepšia sadzba písmena Ľ než v T1
\usepackage[utf8]{inputenc}
\usepackage{graphicx}
\usepackage{url} % príkaz \url na formátovanie URL
\usepackage{hyperref} % odkazy v texte budú aktívne (pri niektorých triedach dokumentov spôsobuje posun textu)

\usepackage{cite}
%\usepackage{times}

\pagestyle{headings}

\title{Podvádzanie v hrách\thanks{Semestrálny projekt v predmete Metódy inžinierskej práce, ak. rok 2022/23, vedenie: Vladimír Mlynarovič}} % meno a priezvisko vyučujúceho na cvičeniach

\author{Patrik Baran\\[2pt]
	{\small Slovenská technická univerzita v Bratislave}\\
	{\small Fakulta informatiky a informačných technológií}\\
	{\small \texttt{xbaranp@stuba.sk}}
	}

\date{\small 14. december 2022} 



\begin{document}

\maketitle

\begin{abstract}
Slovo podvádzanie má v našej spoločnosti negatívny význam. Je na to aj dobrý dôvod, keďže zvyčajne vedie ku neférovej výhode a často býva aj trestné. Pri hrách však ľudia pravidelne vykonávajú činnosti, ktoré nie sú v súlade s pôvodnými zámermi herných dizajnérov.

Najviac príležitostí stretnutia s podvádzaním máme v prípadoch, keď spôsobí neférovú výhodu v hre hranej viacerými ľuďmi naraz. Podvádzanie nie je možné iba v hrách pre viacerých, často prebieha práve keď sme vo svete hry sami.

V rámci tohto článku sa zamýšľame nad rozsahom pojmu podvádzanie, so zameraním na počítačové hry. Potom ako tento rozsah zadefinujeme sa pokúsime poskytnúť dôvody prečo niektorí hráči podvádzajú, vysvetliť rozdiely podvádzania pri hrách pre jedného a hrách pre viac hráčov a vysvetliť ako mu môžu herný vývojári zabrániť.
\end{abstract}

\newpage


\section*{Úvod}
%obsah
%nazov tabulky nad ( vlastna tvorba, upravene podla , prekreslene...

% reakcie na temy prednasok na 3-4
% kazdy paragraf 150 - 200 slov
% namapovat ku teme / zvlast pred literaturou

Podvádzanie je konanie, ktoré v hrách prebieha už od ich samého počiatku. Spôsobov ako podvádzať je mnoho. Nie vždy je však presne zadefinované čo všetko pod tento pojem spadá, teda či záleží na názore ľudí hrajúcich hru, či na názore ľudí ktorí ju vytvorili. V tomto článku sa budeme zamýšľať nad problematikou podvádzania v hrách.

Slovo podvádzanie v našom jazyku nesie mnoho významov, môže znamenať neveru, trik, získanie neférovej výhody. Kým z týchto významov je zrejmé prečo sa podvádzanie chápe ako niečo čisto negatívne, vo svete hier to vôbec nemusí byť tak, nakoľko sa môže odohrať aj mimo akéhokoľvek sociálneho prostredia.

V nasledujucel tabuľke \ref{tabulka} môžeme vidieť počet zabanovaných hráčov v hre Counter Strike: Global Offensive za jednotlivé mesiace roku 2021. Ako vidíme je množstvo ľudí, ktorý porušujú pravidá v hrách, no vo významnom počte prípadov ide o podvádzanie.

\begin{figure}[h]
\caption{tabuľka počtu zabanovaných hráčov v hre Counter Strike: Global Offensive za rok 2021 v tisícoch}
\label{tabulka}
\centering{Zdroj: Vytvorené podľa dát na csgostats.gg \cite{csgo}}
\begin{tabular}{ |c|c|c|c|c|c|c|c|c|c|c|c |}
\hline
Jan. & Feb. & Mar. & Ap5. & Máj & Jún & Júl & Aug. & Sep. & Okt. & Nov. & Dec. \\
\hline
135,9 & 107,4 & 104,6 & 115,3 & 142,1 & 103,6 & 48,2 & 48,7 & 46,2 & 44,2 & 42,0 & 79,2 \\
\hline
\end{tabular}
\end{figure}

V tomto článku najskôr uvedieme problematiku zadefinovania pojmu podvádzanie v čati Hranice\ref{Hranice}. Keď ho zadefinujeme, budeme pokračovať zamyslením sa nad príčinami čo ho vzbudzuje \ref{Motivacie}, a ako by sme mu mohli zabrániť \ref{Opatrenia}.
\section{Hranice}
\label{Hranice}

Problém s presným definovaním pokiaľ siahajú hranice akceptovateľné správanie a odkiaľ sa jedná o podvádzanie spočíva v subjektivite. Je samozrejmé, že každý človek by rád zahrnul svoje správanie ešte pred hranicou podvádzania. To znamená, že vždy záleží na jednotlivých hráčoch,  respektíve komunitách v ktorých hrajú. Náš primárny zdroj podvádzanie zadefinoval ako vonkajšiu hernú výhodu, konanie v rámci hernej mechaniky, no spôsobmi neočakávanými hernými dizajnérmi\cite{mood}.

Spektrum činností, ktoré sa môžu považovať za akceptovateľné v hrách býva rozsiahle. Jedna z dôležitých skupín nad ktorou je potrebné sa zamyslieť sú herný vývojári. Nakoľko práve oni vytvorili hru týmto spôsobom, je možné že budú považovať akúkoľvek výchylku od nimi vytvorenej hry za podvádzanie voči nim. 

Situácia býva veľmi odlišná v takzvaných "speedrun" komunitách. V takýchto komunitách si dávajú hráči za cieľ dokončiť hru v čo najkratšom čase. Vrámci dosiahnutia takéhoto cieľa sa zvyčajne hra delí na kategórie na základe cieľa či pravidiel. Tieto pravidlá bývajú jedinečné naprieč hrami, napríklad Minecraft povoľuje zobrazienie vnútorných údajov, ako aj externé nástroje na čítanie, spracovanie  a vizualizáciu dát priamo z hry. Taktiež tu je "The Legend of Zelda: The Wind Waker", kde sa počas speedrunu vyskytuje časť zameraná na šťastie. Komunita pre túto čásť vyvinula externý nástroj, pomocou ktorého je možné uhádnuť správne usporiadanie na druhý pokus, no za účelom férovosti musia hráči zadávať údaje do tohoto nástroja manuálne\cite{zelda}. Extrémom týchto pravidiel sú hry, pri ktorých sa akceptujú iba časy získané na pôvodnom hardveri.

Jedna z možností ako tento problém vyriešiť je výber charakteristiky podvádzania a následné označenie všetkých aktivít spojených s hraním, ktoré spĺňajú danú charakteristiku, za podvádzanie. Pre účely tohto článku sme sa rozhodli definovať podvádzanie ako každú výhodu, ktorú sa nedá dosiahnuť čisto hraním alebo spôsobom, aký bol pôvodne zamýšľaný v procese vývoja hry.

\subsection{Herné mechaniky}
V hrách sa občas nedopatrením objaví nejaká chyba, ktorá umožni hráčovi prejsť cez pevné objekty, získať neobmedzené množstvo životov, peňazí a podobne. Takéto chyby samozrejme nezvyknú patriť medzi plánované správanie. Nepočítajú sa ako zamýšľaný spôsobu hrania hry, teda podľa našej definície sa jedná o podvádzanie.

Na druhej strane však môže nastať prípad, keď zámerná herná mechanika sa dá využiť v neočakávanej situácii na neočakávanom mieste namiesto predpokladaného riešenia na preskočenie problému. Za týchto okolnosti je možné hovoriť. V takomto prípade si hráč ani len nemusí uvedomiť, že niečo preskočil, keďže sa chyby dopustil samotný vývojár. Napriek tomu však hráč získal výhodu tým, že nehral zmýšľaným spôsobom, takže podvádzal.

\subsection{Pomoc}
V niektorých žánroch hier sa zvyknú vyskytovať nápovedy, ako ďalej postúpiť. Dizajn hry sa však začína vytvorením samotných úrovní, čo je následované otestovaním úrovne a až nakoniec  (ak sa v danej hre vôbec nachádzajú) sa pridajú nápovedy. Tieto nápovedy nie su potrebné na dokončenie hry, pripomínajú skôr nejaký dodatočný nápad po vytvorení úrovne, s výhodou že už nič netreba meniť. Podobná mechanika sa zvykla vyskytovať už v prvých hrách, kde ak hráč zadal určitú postupnosť vstupov, aktivovali sa mu špeciálne schopnosti, poprípade získal prostriedky navyše, neobmedzene životov. Tieto cheaty taktiež boli mechaniky zaimplementované do hry, no považujú sa za podvádzanie. Nápovedy na rozdiel od cheatov nezvyknú byť utajované, no pracujú na podobných princípoch, a tak ich považujeme za podvádzanie.

Návody sú spôsob pomoci, ktoré zvyknú byť vytvorené ďalšími hráčmi. Podobajú sa nápovedám v tom, že poskytnú hráčovi vedomosť, na ktorú mal prísť sám, avšak návody sú dlhšie, a danú vedomosť poskytnú priamo. Jeden z typov návodov je tzv. tutorial, teda úvod do hry, ktorý vás oboznámi so základmi herných mechaník. Takýto úvod do hry je potrebný na samotné hranie, no každý ďaľší návod okráda hráča o príležitosť experimentovať s danými mechanikami, a teda ide o podvádzanie

Posledná možnosť získania pomoci mimo využitia externých programov je požiadanie niekoho iného. Takúto pomoc môžette získať formou nápovede, či formou odohrania danej úrovne. V prípade druhej formy nebudete ovládať hru, čím efektívne preskočíte danú úroveň, a taktiež nezískate skúsenosti potrebne na prejdenie úrovne.

\subsection{Nákupy v hre}\label{nakupy}

Nákupy v hrách sú jedným zo spôsobov ako získať viac fondov na ďalší vývoj hier. Na to aby presvedčili hráča minúť v danej hre peniaze mu môžu ponúknuť viacero produktov. Medzi takéto produkty môže patriť jednoduchá zmena výzoru, pridanie vizuálnych efektov, zmena použivateľského mena, no problematické sú práve ponuky, ktoré vám ponúkaju rôzne herné suroviny alebo výbavu. Týmto umožnujú hráčom kúpiť si spôsob ako pokročiť spôsobom iným ako hraním, či študovaním herných mechaník. Pokročenie v hre minutím peňazí býva dokonca najrýchlejší spôsob ako pokročiť. Takéto konanie je podvádzanie, nakoľko umožňuje premeniť úspech nesúvisiaci s hrou do hry, a je ku tomu ešte aj zlý pre súťaž v hre.

\subsection{Úprava herných súborov}

Na zmenu správania samotnej hry je možné pozmeniť rôzne vnútorné premenné, či pridať rôzne módy, teda úpravy hry ktoré do nej niečo pridajú, alebo zmenia jej správanie. Tieto módy môžu hru uľahčiť, sťažiť, či len jednoducho rozšíriť. Množstvo hier takéto módovanie dokonca aj priamo podporuje. V tomto prípade sa je potrebné zamyslieť či sa nejedná o úplne inú hru/ herný mód, teda či je pozmenenú a pôvodnú verziu možné porovnať s ohľadom na spôsob a ciele hrania.

Jestvuje kategória externého softvéru zameraná práve na uľahčenie hry. Táto kategória sa nazýva hacky. Môžu uľahčovať hru niekoľkými rôznymi spôsobmi, môžu informovať hráča ukazovaním nepriateľov, aj keď stoja za nepriehľadnými objektmi. Môžu vykonávať činnosti namiesto hráča, napríklad zamerať na hlavu súpera. Hacky môžu odstrániť obmedzenia napríklad umožnením hráčovi pohybovať sa rýchlejšie, či lietať. Takéto zmeny počas hry sú samozrejme podvádzanie.

\section{Motivácie pri podádzaní}
\label{Motivacie}

Predchádzajúci výskum objavil 4 hlavné dôvody, prečo sa môže hráč rozhodnúť podvádzať: pocit zaseknutia, túžba zahrať sa na boha, nuda a snaha správať sa ako somár.\cite{Consalvo} Výskumov s podobným cieľom bolo už viacero, pričom dosiahli podobné výsledky, no priamo sa nezhodovali. 

V jednom článku sa spomína porušenie úprimnosti v mobilných hrách. Vysvetľuje, že koncový používatelia majú rôzne očakávania. Očakávajú napríklad súkromie svojich údajov, transparentnosť procesov v aplikácii a etické správanie platform a softvérových spoločností\cite{FirstLook}. Je zrejmé, že porušenie takýchto očakávaní má negatívny dopad na vzťah medzi hráčom a vývojármi tejto hry, a taktiež aj nepriamo na vzťah hráča ku hre\cite{honesty}. Tento článok sa na situáciu zameral z pohľadu vývojárov, pričom spomenul prípady keď softvér podvádzal uživateľa. Tento prípad má veľký súvis s našou témou. Hráči zvyknú konať v rámci povoleného systému, pretože ho rešpektujú. V prípade keď nastane ku porušeniu úprimnosti to môže spôsobiť nedôveru hráča v systém, čiže ku strate rešpektu. Následne môžeme mať pocit, že podvádzaním iba získavame niečo čo nám právom patrí,ak ho to neprinúti nadobro prestať hrať.

Jeden z možných dôvodov teda môže byť aj odveta voči hre samotnej. Samozrejme že dôvodom prečo medzi dôvody podvádzania nespomenuli chýbajúcu dôveru v hru môže byť aj iná definícia podvádzania.

V hrách sa neustále snažíe dosiahnúť nejaký cieľ, či už je definovaný hrou, či nami. Podvádzanie v takomto prípade môže slúžiť na jednoduché dosiahnutie pocitu zadosťučinenia. No v prípadoch, keď každý hráč nie je izolovaný vo svojom svete a hráči medzi sebou aktívne súťažia, môže takáto príležitosť zmariť očakávanie férového súperenia. Najmä v prípadoch kde náhoda hrá svoju rolu môže hráč získať presvedčenie, že si zaslúži byť na vrchole rebríčka, čo znova môže viesť ku vyhľadávaniu zakázanej výhody nad ostatnými.

Ďalšou možnosťou môže byť nejaká snaha o férovosť. Niektoré kategórie nákupov v hier sme zahrnuli v našej definícii podvádzania\ref{nakupy}. Jedným zo spôsobov, ako nám hra môže ponúknuť spôsob zakúpenia výhody je nákup špeciálnej meny, ktorou je možné si zakúpiť výhody. Jedným zo zaužívaných spôsoboj je ponuka vyhrať ceny v rulete, do ktorej sa dajú zakúpiť ďalšie pokusy. Nejaké obmedzené množstvo zatočení ruletov môžu byť prístupné pre všetkých hráčov, no už podľa samotnej definície založené na náhode. Toto preddstavuje relatívne nový problém hier. Keďže používajú vývojári náhodu, hráč môže v snahe zosilnieť nájsť odôvodnenie, prečo si zaslúži mať silnejšiu postavu už teraz. Niektorý hráči sa aj môžu stáť zberateľmi takýchto odmien. V následujúcom diagrame \ref{diagram} môžeme vidieť proces získavania výbavy v hre Genshin Impact. Nejaké malé množstvo zatočení je možné získať samotným hraním, no ako môžeme vidieť, šanca na získanie konkrétneho 5* vybavenia je celkom nízka. Čo tento jav len naďalej zhoršuje je fakt, že tieto postavy bývajú časovo obmedzené.

\begin{figure}[h]
\caption{diagram procesu točenia pre výbavu}
\label{diagram}
\centering{Vytvorený podľa informácii z \cite{wish}}
\includegraphics[scale=0.34]{wish_diagram.pdf}
\end{figure}

\section{Opatrenia proti podvádzaniu}
\label{Opatrenia}

Množstvo online hier v dnešnej dobe používa rôzne systémy na detekciu podvádzania, ktoré sa spúšťajú priamo s hrou. V dnešnej dobe je dokonca aj trend vyžadovať internetové spojenie aj v hre pre jedného na overenie hry. Dôvodom na tekéto overovanie býva vlastníctvo hry, no je to jedna z činností, ktoré zneviditeľnujú rozdieli medzi hraním online a offline.

Jasný spôsob ako odradiť hráčov od podvádzaní sú tresty. Tento spôsob je v dnešnej dobe celkom rozšírený, zvyčajne sa jedná o zákaz hrania na určitú dobu, až úplný zákaz. Jedná sa o formu inšpirovanú ozajstnými zákonmi, no bežne jestvujú spôsoby ako sa takémuto zákazu vyhnúť, nakoľko je relatíne problematické zistiť či sa hráč náhodou nepripojil z iného zariadenia na inom mieste, ak sa rozhodne pripojiť cez iné konto.V predchádzajúcej části sme už spomenuli dôležitosť etiky v tejto téme, a samozrejme aj tresty by mali byť morálne. V princípe takéto zákazy znemožnujú používanie zakúpených služieb. Takéto taktiky sa síce vyskytujú v zákonoch, možno existujú iné spôsoby snáď lepšie než úplné zamietnutie práva hrať hru.

Jeden zo spôsobov ako odradiť hráčov od podvádzania môžu byť aj neočakávané dôsledky. 

Touto témou sa zaoberajú aj niektoré štúdie. Jedna takáto štúdia sa zamerala na efekt sledujúcich očí. Prišla s hypotézou, že virtuálna postava spôsobí pocit že nás niekto sleduje, teda starosť o našu reputáciu, čo znamená, že sa budeme väčšmi chcieť byť úprimný. \cite{not_alone} Túto hypotézu sa jej dokonca aj podarilo potvrdiť.

Najlepším spôsobom prevencie však bude odstránenie dôvodov. Pred tým než sa pokúsime odstrániť všetky dôvody podvádzania je ale potrebné sa zamyslieť, či podvádzanie predsa len nemá nejaké dobré účinky. Štúdia zameraná na emočný vplyv podvádzania prišla s výsledkom, že to vo väčšine prípadov aj ten pozitívny vplyv má. Jedna z najväčších prekážok tomuto pozitívnemu vplyvu je ale hanba ako efekt negatívneho vnímania podvádzania v spoločnosti\cite{mood}. Mali by sme sa teda uistiť, že odstraňujeme podvádzanie s negatívnym vplyvom.

Možnosti ako predísť podvádzaniu je viacero, no líšia sa oblastiami podvádzania ktoré ovplyvňujú. Zneužitiu herných mechaník sa dá predísť zvýšenou pozornosťou počas vývoja, a dlhšou dobou testovania, ako aj aktualizáciami po vydaní. V prípadoch keď hráč vyhľadáva pomoc, môže byť príčinou nepochopenie obtiažnosti hry pred kúpou, alebo nesprávne vysvetlenie mechaník, príliš unáhlené zakomponovanie viacerých mechaník. V prípade herných nákupov ako jediné riešenie vidíme odstránenie nákupov ponúkajúcich rýchli pokrok. Využitie externých súborov počas hier sa dá odhaliť, no žiaľ nie vo všetkých prípadoch. Sme názoru, že tomuto aspektu podvádzania nebude možné úplne zabrániť, pokiaľ budú hráči môcť hrať vo svojích domácnostiach na svojích zariadeniach.

\section*{Vyjadrenie ku témam prednášok}

\paragraph{Grafické vyjadrenie informácií}
Evolučne sme zameraný skôr na zapamätávanie si obrázkov, tvarov, hlasov a chutí, než na text. Text môže byť na takéto informácie hustejší, no existuje množstvo príslový typu "Je lepšie raz vidieť ako stokrát počuť", a bude za tým aj nejaký ten kus pravdy. Taktiež nám určite pomôže si zapamätať informáciu ak pri jej získani získame viac vnemov, použijeme viac zmyslov.
Grafické vyjadrenie informácií je jeden zo spúsobov, ako ľahko podať ďalej informáciu. Určite to bude aj najlepší spôsob na vysvetlovanie, koniec koncov už pri počúvaní reči, či čítaní textu si potrebujeme niekoľko vecí predstaviť na to aby sme mu mohli porozumieť. Takisto to platí pri vytváraní konceptov, je omnoho jednoduchšie niečo popísať obrázkom, a neskôr vykonať mierne úpravy, než musieť prepisovať celé paragrafy pri dokumentácii. Na tento účel nám môžu poslúžiť diagramy.

\paragraph{Technológia a ľudia}
Priamo v úvode prednášky ma zaujala snímka ktorá definovala projekt ako organizované úsilie za účelom naplnenia ideí. Tento výrok bol nepochybne myslený na tímové projekty, no vlastne aj náš vlastný článok je taký projekt. Na začiatku každého z naších článkov musel byť nejaký nápad, ktorý bol náš vlastný, no aby sme ho mohli zrealizovat potrebovali sme prácu ľudí, ktorí prišli pred nami.
V rámci dotazu o premýšľaní pred začiatkom realizácie projektu nerád súhlasím práve s jednou možnosťou. V ideálnom prípade, keď sme si vedomí všetkých súvislostí potrebných na zrealizovanie projektu by sme si mali premyslieť iba to najnutnejšie. V reálnom sveet však s ideálmi nepočítame. Preto najskôr potrebujeme začať robiť, otestovať naše nápady v malom, a až keď vieme že budú fungovať si vieme premyslieť plán. Chápem, že na "správnu" sme prišli počas zvyšku prednášky, ktorá vlastne spojila výhody čiastočného plánovania a okamžitého začiatku implementácia vďaka viacerým iteráciam, kde sa části prelínajú.
Chcel by som sa vyjadriť aj ku vykonávaniu rozhodnutia v zodpovedno momente namiesto najzodpovednejšieho. Týmto rokom som začal prvý krát za svoj život žit samostatne. Nič ma nenúti vykonávať povinnosti, a tak je ľahké ich odkladať za zámienkou lepšej chvíle, keď budem viac motivovaný. Lenže už takéto správanie už nie je možné, je kľúčové naučiť sa neodkladať rozhodnutia na neskôr, prioritizovať funkčnosť a samozrejme robiť čo je požadované, skôr toho čo sa nám chce.

\paragraph{Udržateľnosť a etika}
Za ľudským konaním vždy stojí nejaký dôvod. Keď hráme hry zvyčajne je cieľom dosiahnúť nejaký stav oddychu alebo potešenia. Pri prednáškach sa zase snažíme niečo naučiť, najmä ak sa jedná o tému, ktorá je nám blízka. Samozrejme že za všetkým konaním stojí taktiež nejaká snaha, presvedčenie, teda etické a morálne vplyvy. Toto však nestačí na udržanie návštevnosti, či počtu aktívnych hráčov.
Z hľadiska témy podvádzania v hrách sú silné stránky pravdepodobne články rozoberajúce túto tématiku. Na úvod som očakával niekoľko vyjadrení k témam, no nakoniec som objavil množstvo článkov, ktoré ak nesúviseli s podvádzaním priamo, tak sa zameriavali na blízke témy. Slabosťou témy môže byť náročnosť s uspokojivým definovaním podvádzania. Možnou príležistoťou pre rozmach tejto témy by určite bolo viac konverzácie, no môže to byť dvojsečná zbraň, ak budeme nadaľej túto tému iba zahanbovať.
Z etického hľadiska je táto téma jasne že problematická. V rámci túžbi zahrať sa na boha, či byť zlý sa hráči zvyknú správať neeticky voči ostatným. Určite by sa taktiež aj hodilo lepšie vyriešiť tresty za podvádzanie, nakoľko sa ním momentálne dá vyhnúť.

\section{Záver} 

Najväčším problémom tejto témy je jej subjektívnosť. Nie je možné definovať podvádzanie jednoducho ako správanie vymykajúce sa zámerom herných dizajnérov. Definovanie ako získanie neférovej výhody však taktiež nie je postačujúce, keď samotný systém hry poskytuje ostatným neférové výhody, alebo ich má nad vami. Aj keby sa to niekomu podarilo objektívne definovať podvádzanie, po niekoľkých rokoch by sa zmenilo vnímanie danej hry a definícia by prestala platiť, respektíve by prestala byť úplná.

Aj napriek nášmu záveru avšak existujú rôzne iné štúdie. Napríklad náš primárny zdroj, zameraný na emočný vplyv podvádzania, pracovala so 4 rôznymi cieľmi podvádzania, a to "\ oprava nálady", úľava od stresu, naplnenie potrieb a optimalizácie stavu vnorenia do hry\cite{mood}. 

Dokonca aj hry pre jedného v dnešnej dobe zvyknú zaznamenávať určitý zoznam úspechov, čo informuje každého kto sa pozrie na náš profil o našich úspechoch, a teda aj do týchto hier prináša dilemy podvádzania v hrách pre viacerých hráčov.

Koniec koncov vždy záleží od danej komunity. Ak sa rozhodne, že je niečo povolené bez súhlasu väčšíny, väčšina si môže založiť vlastnú frakciu s inými pravidlami.


\nocite{*}

% týmto sa generuje zoznam literatúry z obsahu súboru literatura.bib podľa toho, na čo sa v článku odkazujete
\bibliography{literatura}
\bibliographystyle{plain}
\end{document}
